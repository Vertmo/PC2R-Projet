\documentclass{article}
\usepackage{listings}
\lstset{language=Caml}

\title{PC2R - Projet - Vector Arena}
\author{Basile Pesin\\Sorbonne Université}

\begin{document}
\maketitle


\section{Structure du projet}

\subsection{Choix techniques}
On a choisi de réaliser le serveur en Java: ce choix nous permet de facilement établir la partie ``serveur'' proprement dite (au moyen de la classe \textit{ServerSocket}. De plus, les classes Java nous permettent de facilement encapsuler et protéger les données de jeu.\\
D'autre part, pour implémenter le client, on a choisi d'utiliser le langage OCaml, pour la simplicité d'utilisation de ces threads (et puisque, comme on le verra, le client utilise un certain nombre de threads différents). La petite difficulté aurait put être la réalisation d'une interface graphique dans ce langage, heureusement la bibliothèque \textit{Graphics} a permit de réaliser une interface très suffisante (bien que simple).

\subsection{Utilisation}
Les dépendances logicielles sont les suivantes:
\begin{itemize}
  \item \textbf{java} (le programme a été testé avec les versions 10 et 11)
  \item \textbf{apache ant}
  \item \textbf{OCaml v4.06.1} et \textbf{opam}
  \item \textbf{dune}: peut être installé avec \textbf{opam install dune}
  \item La bibliothèque \textbf{Graphics} d'OCaml (cela peut poser quelques problèmes sur Mac...)
\end{itemize}
Le serveur peut être compilé et exécuté en lançant la commande\\
\centerline{\textbf{ant run}}\\
dans le dossier \textit{server/}. Le client peut lui être compilé avec\\
\centerline{\textbf{dune build src/pc2rClient.exe}}\\
puis lancé avec\\
\centerline{\textbf{./\_build/default/src/pc2rClient.exe -a localhost -p 45678 riri}}\\
dans le dossier \textit{client/} (le serveur doit être lancé avant le client). Bien sur, plusieurs clients peuvent être lançés en même temps, y compris sur la même machine.\\\\
Une fois au moins un joueur connecté, le serveur attends 10 secondes (20 secondes semblait trop long) puis démarre une session. Comme le jeu a été développé sur un ordinateur avec un clavier qwerty, les commandes sont les suivantes:
\begin{itemize}
\item \textbf{w} pour thrust
\item \textbf{d} pour clock
\item \textbf{a} pour anticlock
\end{itemize}
Le joueur doit alors attraper l'objectif. Le premier joueur à obtenir 5 points gagne ! (avant les extensions ou l'on modifiera les rêgles).

\section{Réalisation}

\subsection{Partie A}

\subsubsection{Client}
Comme on l'a précisé plus haut, le client a été réalisé en OCaml, en utilisant la bibliothèque \textbf{Graphics}. Le client exécute 4 threads en parallèle (pour autant que les threads OCaml soient réellement parallèles):
\begin{itemize}
  \item Le thread principal, après avoir lancé les 3 autres threads, s'occupe principalement de gérer le mouvement du joueur, c'est à dire de mettre à jour la position du jour en fonction de sa vitesse. Le thread boucle donc en se mettant en pause pendant $\frac{1}{refresh\_tickrate}$ (on a réglé $refresh\_tickrate = 60$) au moyen de la fonction $Thread.delay$.
  \item Le thread listener écoute les communications du serveur sur la socket ouverte au début de son execution. La fonction permettant de lire la socket est bloquante tant que celle ci ne reçoit rien, et ce thread ne fait donc pas d'attente active. Ce thread exécute ensuite les commandes reçues depuis le serveur, principalement en modifiant l'état global du jeu (qui est protégé en écriture par un $Mutex.t$).
  \item Le premier des deux threads gérant l'interface s'occupe de mettre à jour l'interface graphique (c'est à dire de redessiner la position des véhicules, de l'objectif...). Ce thread (et donc l'affichage) est rafraichie à la même fréquence que le thread principal (c'est à dire $refresh\_tickrate$).
  \item Le dernier thread est chargé de prendre en compte les entrées du joueur (pression sur les touches), au moyen de la fonction $Graphics.read\_key$, qui est bloquante. Ce thread attends donc une entrée du joueur, et exécute la commande correspondante (le thread ne fait donc pas d'attente active).
\end{itemize}

\subsubsection{Serveur}
Le serveur est encore plus multi-threadé que le client, puisque, comme on le verra, chaque client dispose d'un thread à son écoute. En effet, le serveur possède une $ServerSocket$, sur laquelle il accepte les connections des clients; il traite ensuite chacune de ces connections dans un nouveau thread, chargé d'attendre les commandes du client et des les éxècuter. Comme pour le client, la lecture de la socket de communication est bien sure bloquante. Une fois une commande reçue, elle est parsée puis exécutée (les commandes reçues des clients sont implémentées selon le design pattern ``Command'', ce qui produit en général un effet sur l'état du jeu tel que stocké côté serveur (dans la classe $pc2r.game.GameState$, dont les écritures sont synchronisées).\\
En plus de ces threads chargés de la communication, le serveur dispose aussi du thread implémenté dans la classe $Game$, gérant la boucle principale de jeu. Après avoir attendu la connection d'au moins un joueur (attente non active au moyen d'une $Condition$ qu'on signale à la connection), la boucle envoie les mises à jour des positions aux clients tous les $\frac{1}{server\_tickrate}$ (on a fixé $server\_tickrate = 6$) au moyen de la commande $TICK$ (ce à quoi les client répondront par un $NEWPOS$, qui sera traité non pas par ce thread mais par leurs threads de communication respectifs). Le serveur vérifie aussi si un des clients à atteint l'objectif, et le cas échéant déclare un point et/ou la fin de la session.

\subsection{Partie B}

\subsection{Serveur}
Le déplacement des calculs de déplacement coté serveur nécessite quelques ajustements sur celui-ci. Entre autre, le thread $Game$, est maintenant simplement chargé des calculs de déplacement et de la détection de la rencontre avec l'objectif, et est rafraichi à intervalle $frac{1}{server\_refresh\_tickrate}$ (qu'on a réglé à la même valeur que $refresh\_tickrate$, c'est à dire $60$, pour que les déplacements calculés par le serveur et par les clients correspondent). Le serveur attends donc l'arrivée de commandes $NEWCOM$ sur la socket de chaque client, suite à quoi il répond au moyen d'une commande $TICK$.\\
Il est important de noter que cette version du serveur, en plus de ne plus accepter la commande $NEWPOS$ (ne rien faire à sa reception) modifie aussi la syntaxe de la commande $TICK$. Un serveur en version ``Partie B'' ne sera donc pas compatible avec un client ``Partie A'' (et le contraire est vrai aussi, puisqu'un version ``Partie A`` du serveur n'implémenterait pas la commande $NEWCOM$).

\subsection{Client}
Le client lui, nécessite l'ajout d'un nouveau thread, uniquement chargé d'envoyer les noubelles commandes $NEWCOM$ (et pas des $NEWPOS$). C'est ce nouveau thread ($Connection.commands\_sender$) qui est rafraichie à la fréquence $\frac{1}{server\_tickrate}$. L'utilisation de commandes $NEWCOM$ signifie aussi que les commandes entrées par l'utilisateur sont stocker en attendant d'étre envoyées au serveur, plutôt qu'exécutées directement.\\
Petite amélioration par rapport à la version précédante: comme la commande $TICK$ transmet maintenant au client la direction des véhicules ennemis, on peut maintenant l'afficher à l'écran, ce qu'on ne pouvait faire dans la version précédante.

\subsection{Observations}
Comme on l'a vu plus haut, nous avons réglé $server\_tickrate = \frac{refresh\_tickrate}{10} = \frac{server\_refresh\_tickrate}{10}$. Avec ces paramêtres, le jeu reste assez jouable, mais est assez désagréable visuellement, puisque les véhicules semblent en effet se téléporter de quelques pixels à chaque $server\_tickrate$. Le temps de réaction après l'appui sur une touche rends aussi le jeu moins agréable à jouer. On tentera d'améliorer ce problème dans l'une des extensions (calcul coté client et vérification).

\subsection{Partie C}
Cette partie ne nécessite que peu de modifications. Comme indiqué dans l'énoncé, on modifie les commandes $WELCOME$ et $SESSION$ pour indiquer la position des obstacles, ce qui rend un serveur/client en version ``Partie C'' incompatible avec un client/serveur en version ``Partie B''.\\
D'un point de vue serveur, le calcul des collisions est réalisé en même temps que les calculs de mouvements par le thread $Game$. D'un point de vue du client, on ajoute la visualisation des obstacles à leurs coordonnées respectives.

\section{Extensions}

\subsection{Calcul coté client et vérification}
Comme on l'a vu plus haut, le ``lag'' crée par les calculs côté serveur est une assez grande source de frustration, puisque le jeu en deviens beaucoup moins jouable (et par ailleurs moins agréable visuellement). On a donc décidé, avant d'ajouter de nouvelles fonctionnalités au jeu, de commencer par régler ce problème.\\
Pour ce faire, on a ajouté une nouvelle commande envoyant à la fois les coordonnées actuelles du véhicule du joueur, et la dernière commande entrée. Cette commande (Client $\rightarrow$ Serveur) a la syntaxe suivante:
$$NEWCOMPOS/coord/comms/$$
avec $coord$ et $comms$ telles que définies dans le sujet.\\
A la reception de cette commande, le serveur met à jour les informations (vitesse et angle) du joueur, comme il le ferait pour la commande $NEWCOM$. Puis, le serveur vérifie si les coordonnées reçues sont suffisemment proche de celles qu'il avait calculé préalablement (ici on a pris comme distance maximale une quinzaine de pixels). Si c'est le cas, il accepte les coordonnées calculées par le client.\\
On voit en jouant que cette extension permet bien de réduire les ``téléportations'' du véhicule. Cependant, celles-ci se manifestent encore quand le véhicule se déplace à grand vitesse (puisque les écarts entre les calculs du client et du serveur sont alors potentiellement plus importants). On pourrait absorber ses écarts en augmentant la distance maximale autorisée, ou bien en faisant de cette distance une fonction croissante de la vitesse du véhicule. Cependant, si on suivait cette dernière piste, le client pourrait alors tricher en communiquant une vitesse très importante (ce qui augmenterait l'écart maximal accepté par le serveur), puis en communiquant les coordonnées de l'objectif. On choisit donc de conserver le compromis actuel.

\subsection{Jeu de course}
Pour implémenter le jeu de course, ou plusieurs objectifs sont présents à la fois à l'écran et doivent être traversé dans l'ordre, on modifie les commandes $WELCOME$ et $SESSION$ de la manière suivante:
$$WELCOME/phase/scores/coords/coords/$$
$$SESSION/coords/coords/coords/$$
Où, dans les deux cas, le nouveau $coords$ contient maintenant la liste des coordonnées des objectifs. On remarque qu'à priori un client non étendu sera compatible avec ces instructions si le nombre de coordonnées de la liste est 1.\\
Par souci d'économies on réutilise $win\_cap$ (qui est ici réglé à 5) pour donner le nombre d'objectifs. Les joueurs devront donc passer les 5 objectifs pour gagner la course. Pour détecter la progression d'un joueur, le serveur ne compare sa position qu'avec le prochain objectif qu'il doit traverser (le numéro du prochain objectif à traverser est égal au score du joueur).\\
Du coté client, on prends en compte ces changements en affichant l'ensemble des objectifs (avec un numéro et une décoration supplémentaire sur l'objectif actuel pour aider le joueur à se repérer). De plus, on arrête de prendre en compte la commande $NEWOBJ$ (a part pour mettre à jour les scores) puisqu'elle n'a plus vraiment de sens. On remarque tout de même que comme on n'a pas modifié cette instruction côté serveur, on pourrait la aussi utiliser des clients non mis à jour avec.\\
Un serveur sans cette extension peut sans problème être utilisé avec un client utilisant l'extension (le client ne verra alors qu'un seul objectif). Comme on l'a vu, dans le sens contraire, les clients sans cette extension peuvent être utilisé avec un serveur étendu, mais seulement si le nombre d'objectifs $win\_cap = 1$.

\subsection{Jeu de combat}
Dernière extension implémentée, on décide de donner des armes destructrices à nos joueurs ! Leur fonctionnement est le suivant : le joueur peut tirer en appuyant sur la touche \textbf{e}. Un laser part alors en ligne droite, dans la direction pointées par le véhicule du joueur au moment du tir. Le joueur peut tirer autant de laser qu'il veut, mais il y a un piège: les lasers prennent en compte la forme torique de l'arène, et peuvent en plus toucher leur tireur: il est donc risquer d'``arroser'' de lasers, puisqu'on risque de se toucher soit même. Les lasers continuent donc d'avancer à l'infini, jusqu'à rencontrer un obstacle (auquel cas ils disparaissent) ou un joueur: dans ce second cas, le joueur est stoppé et est incapacité pendant 2 secondes (le serveur ne réagit plus à ses commandes).\\
Afin d'implémenter cette extension, on a ajouté trois nouvelles commandes au protocoles. La première est une commande Client $\rightarrow$ Serveur qui permet d'enregistrer un nouveau tir:
$$NEWBULLET/coord/speed/angle/$$
Cette commande est envoyée en même temps que les commandes $NEWPOSCOM$ de l'extension précédante, c'est à dire tous les $server\_tickrate$. Le client ne peut donc tirer qu'un laser par $server\_tickrate$, ce qui limite un peu le chaos de la partie.\\
D'autre part, on ajoute les deux commandes Serveur $\rightarrow$ Client suivantes:
$$BULLETTICK/vcoords/$$
$$STUN/time/$$
$BULLETTICK$ met à jour les positions des lasers chez les clients (principalement pour affichage). D'autre parts, la commande $STUN$ informe le client qu'il a été touché par un laser, et sera donc incapacité pendant $TIME$ millisecondes ($TIME$ est un flottant).\\
Comme on le constate en jouant, les lasers montrent un peu de ``lag'' durant leur déplacement: c'est bien sur dù à la différence entre le $server\_tickrate$ et le $refresh\_tickrate$ qui rend la prédiction imparfaite. On ne peut malheureusement pas régler ce problème comme on l'avait fait dans l'extension \textit{Calcul coté client et vérification}, puisque c'est le serveur qui gère les déplacement des lasers.

\end{document}
